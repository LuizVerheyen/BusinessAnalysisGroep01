%===============================================================================
% LaTeX sjabloon voor de bachelorproef toegepaste informatica aan HOGENT
%===============================================================================

\documentclass[dutch,dit,thesis]{hogentreport}

\usepackage{lipsum} % Voor voorbeeldtekst, kan later verwijderd worden
\usepackage{graphicx} % Voor afbeeldingen
\graphicspath{{../graphics/}}

% Voor broncode-highlighting
\usepackage[chapter]{minted} % outputdir niet meer nodig
\usemintedstyle{solarized-light}
\setminted{
    autogobble,
    frame=lines,
    breaklines,
    linenos,
    tabsize=4
}

% Lijst van listings in TOC
\renewcommand\listoflistingscaption{%
    \IfLanguageName{dutch}{Lijst van codefragmenten}{List of listings}
}
\renewcommand\listingscaption{%
    \IfLanguageName{dutch}{Codefragment}{Listing}
}
\renewcommand*\listoflistings{%
    \cleardoublepage\phantomsection\addcontentsline{toc}{chapter}{\listoflistingscaption}%
    \listof{listing}{\listoflistingscaption}%
}

% Bibliografie
\addbibresource{bachproef.bib}
\addbibresource{../voorstel/voorstel.bib}
\defbibheading{bibempty}{}

\renewcommand{\cleardoublepage}{\clearpage}
\renewcommand{\arraystretch}{1.2}

%% Document metadata
\author{Luiz Verheyen}
\supervisor{Gilles Blondeel}
\title[Optionele ondertitel]{Titel van de bachelorproef}
\academicyear{\advance\year by -1 \the\year--\advance\year by 1 \the\year}
\examperiod{1}
\degreesought{\IfLanguageName{dutch}{Professionele bachelor in de toegepaste informatica}{Bachelor of applied computer science}}
\partialthesis{false}

%% Begin document
\begin{document}

%---------- Front matter -------------------------------------------------------
\frontmatter
\maketitle

%---------- Inhoudsopgave, lijsten --------------------------------------------
\tableofcontents
\listoffigures
\listoftables
\listoflistings

%---------- Kern ---------------------------------------------------------------
\mainmatter{}

%------ Voorblad informatie ------
\chapter*{Voorblad informatie}
\begin{itemize}
    \item Klas: ... 
    \item Groepsnummer: ... 
    \item Groepsleden: ...
    \item Onderwerp casus: ...
\end{itemize}

%------ Literatuurstudie ------
\chapter{Literatuurstudie}
\lipsum[1-2]
% TODO: Voeg samenvatting van theorie toe
% TODO: Formuleer eigen hypothese volgens de literatuur

%------ Vragen voor het interview ------
\chapter{Vragen voor het interview}
\lipsum[3]
% TODO: Voeg hier de vragen voor het interview toe

%------ Analyse van het interview ------
\chapter{Analyse van het interview}
\lipsum[4]
% TODO: Welke antwoorden kwamen er op de vragen
% TODO: Vergelijk antwoorden met theorie

%------ Reflectie ------
\chapter{Reflectie}
\lipsum[5]
% TODO: Beschrijf het verzamelen van informatie en eventuele weerstand

%------ Bronvermelding ------
\chapter*{Bronvermelding}
% TODO: Voeg hier literatuur en AI-vermeldingen volgens APA 7 toe
% Voorbeeld AI-vermelding:
% OpenAI. (2025). ChatGPT [Large language model]. https://chat.openai.com/

%---------- Bijlagen -----------------------------------------------------------
\appendix

\chapter{Onderzoeksvoorstel}
Het onderwerp van deze bachelorproef is gebaseerd op een onderzoeksvoorstel dat vooraf werd beoordeeld door de promotor. Hier kan je het voorstel opnemen of samenvatten.

%---------- Backmatter, referentielijst ---------------------------------------
\backmatter{}
\setlength\bibitemsep{2pt}
\printbibliography[heading=bibintoc]

\end{document}
